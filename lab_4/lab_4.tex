\documentclass[11pt, letterpaper, includehead]{article}

%%%%%%%%%%%%%%%%%%%%% Pre-document %%%%%%%%%%%%%%%%%%%%%
\usepackage{fancyhdr}  % Allow for headers
\usepackage{graphicx}  % Allow for figures 
\usepackage{float}     % Allow for figure inserted in specified location
\usepackage{amsmath}   % Allow for aligned math
\usepackage{array}     % Allow for cell width manipulation
\usepackage{nicematrix}
\usepackage{amssymb} % Uhhhh what was this????
\usepackage{multicol}

\setlength{\parindent}{0pt} % Remove auto paragraph indents

% Get rid of those big ass margins
\usepackage[margin=1in]{geometry}

% Table cell formatting
\setlength{\arrayrulewidth}{0.25mm}
\setlength{\tabcolsep}{11pt}
\renewcommand{\arraystretch}{1.2}

\begin{document}


Find x velocity for tests

$$\Delta x = V_{0_x}t + \frac{1}{2}a_xt^2$$
$$\Delta x = V_{0_x}t$$
$$V_{0_x} = \frac{\Delta x}{t}$$

%%%%%%%%%%%%%%%%%%%%% Title Page %%%%%%%%%%%%%%%%%%%%%
\begin{titlepage}
  \begin{center}
    \Huge{\textbf{Lab 4}}\\
    \Huge{Projectile Motion}
    \vfill
    \begin{figure}[H] % H makes the figure insert at the position in the document
      \centering 
      \includegraphics[width=6cm]{../logo.png}
    \end{figure}
    \large{\textbf{your name here}}\\
    \large{Julian Barossi, Liam Gilligan, Stephanie L'Heureux}\\
    \vspace{0.5cm}
    \normalsize
    \today
  \end{center}
\end{titlepage}

%%%%%%%%%%%%%%%%%%%%% TABLE OF CONTENTS %%%%%%%%%%%%%%%%%%%%%
\tableofcontents
\pagebreak % Move to next page

% Add a nice fancy header
\pagestyle{fancy}
\fancyhead{}
\fancyhead[C]{\textbf{Lab 4:} Projectile Motion}

\section{Measuring the launch Speed (Horizontal Projections)} % 1

\subsection{Measurements} % 1.1

\subsubsection{Verticle distance} % 1.1.1
The distance $\Delta y$ for the ball to fall from the table was measured in two parts. 
First we measured the height from the bottom of the ball to the bottom of the launcher on the lab table.
This distance was found to be $24.20cm$. Then the distance from the edge of the lab table to the 
floor was $93.70cm$. Combining these we get $\Delta y$.
$$\Delta y = 24.20cm + 93.70cm = 117.90cm \Rightarrow 1.18m$$ 
$$\boxed{\Delta y = 1.18m}$$ 

\subsubsection{Horizontal distance} % 1.1.2
We fired the ball from the lab table at a launch angle of $\theta = 0^{\circ}$ and recorded the $\Delta x$ distance.
$$\Delta x = 349.25cm \approx 3.49m$$
$$\boxed{\Delta x = 3.49m}$$

\subsection{Calculations} % 1.2
\begin{multicols}{3}
  \centering{\textbf{Time in air (t)}}
  $$\Delta y = V_{0_y}t + \frac{1}{2}a_yt^2$$
  $$\Delta y = \frac{1}{2}a_yt^2$$
  $$t = \sqrt{\frac{2\Delta y}{a_y}}$$
  $$t = \sqrt{\frac{2(1.18m)}{9.8 m/s^2}}$$
  $$\boxed{t = 0.491s}$$
  
  \columnbreak
  
  \centering{\boldmath{\textbf{Horizontal velocity} ($V_x$)}}
  $$\Delta x = V_{0_x}t + \frac{1}{2}a_xt^2$$
  $$\Delta x = V_{0_x}t$$
  $$V_{0_x} = \frac{\Delta x}{t}$$
  $$V_{0_x} = \frac{3.4925m}{0.491s}$$
  $$V_{0_x} = 7.113 \, m/s \approx 7.11\,m/s$$
  $$\boxed{V_{0_x} = 7.11\,m/s}$$

  \columnbreak

  \centering{\boldmath{\textbf{Launch speed} ($V$)}}
  $$V = \sqrt{V_x^2 + V_y^2}$$
  $$V = \sqrt{(7.113 \, m/s)^2 + (0 \, m/s)^2}$$
  $$V = 7.113 \, m/s \approx 7.11 \, m/s$$
  $$\boxed{V = 7.11 \, m/s}$$
\end{multicols}

\subsection{Average launch speed, standard deviation, and a standard error} % 1.3

\subsection{Best value for initial speed} % 1.4
\section {Measuring the launch speed (Vertical Projections)} % 2
\subsection{Max y height reached} % 2.1
\subsection{Launch speed} % 2.2
By plugging in the displacement from the end of the spring
in the launcher to the maximum height into $\Delta y$, we
can calculate the launch speed for that measurement.

$$V_y^2 = V_{0_y}^2 + 2a_y\Delta y$$
$$0 = V_{0_y}^2 + 2a_y\Delta y$$
$$-V_{0_y}^2 = 2a_y\Delta y$$
$$V_{0_y}^2 = -2a_y\Delta y$$
$$V_{0_y} = \sqrt{-2a_y\Delta y}$$
\subsection{Average launch speed, standard deviation, and a standard error} % 2.3
\subsection{Best value for initial speed} % 1.4

\section{Comparing your Results} % 3
\textbf{Which is most precise? Does this mean that it is the most accurate?}\\ 

\textbf{Are the measured vertical and horizontal launch speeds the same or is there a
significant difference between them? Should they be identical in theory? Why or
why not?}\\

\textbf{What value for initial speed should you use for making predictions in part 4, and
why?}

\section{Challenge Shot} % 4 
\subsection{Prediction} % 4.1
\subsubsection{Optimal theta} % 4.1.1
Set maximum height to be average of top pole and and bottom pole, find $V_{0_y}$ and therefore $\theta$,
as $V_{0_y} = V_0\cdot\sin\theta$

$$V_y^2 = V_{0_y}^2 + 2a_y\Delta y$$
$$0 = V_{0_y}^2 + 2a_y\Delta y$$
$$-V_{0_y}^2 = 2a_y\Delta y$$
$$V_{0_y}^2 = -2a_y\Delta y$$
$$V_{0_y} = \sqrt{-2a_y\Delta y}$$
$$V_0\cdot\sin\theta = \sqrt{-2a_y\Delta y}$$
$$\sin\theta = \frac{\sqrt{-2a_y\Delta y}}{V_0}$$
$$\theta = \arcsin\left(\frac{\sqrt{-2a_y\Delta y}}{V_0}\right)$$
$$\theta = \arcsin\left(\frac{\sqrt{-2(-9.80m/s^2)(0.4845m)}}{7.03m/s}\right)$$
$$\theta = \arcsin(0.4377)$$
$$\theta = 26^{\circ}$$
\subsubsection{Time to maximum height} % 4.1.2
$$V_y = V_{0_y} + a_yt$$
$$0 = V_{0_y} + a_yt$$
$$-V_{0_y} = a_yt$$
$$t = \frac{-V_{0_y}}{a_y}$$
$$t = \frac{-7.03m/s\cdot\sin26^{\circ}}{-9.80m/s^2}$$
$$t = 0.314441s \approx 0.31s$$
\subsubsection{Horizontal distance} % 4.1.3
$$\Delta x = V_{0_x}t + \frac{1}{2}a_xt$$
$$\Delta x = V_{0_x}t$$
$$\Delta x = 7.03m/s\cdot \cos26^{\circ}\cdot 0.31s$$
$$\Delta x = 1.98666m \approx 2m$$
\subsection{Analysis}  % 4.2
% How accurate was the prediction 

\subsection{Recorded time} % 1.2 
\begin{center}
  \begin{tabular}[H]{| m{2cm} | m{2cm} |}
    \hline
    \textbf{Trial} & \textbf{Time (s)} \\
    \hline
    1              & 0.55              \\
    \hline
    2              & 0.69              \\
    \hline
    3              & 0.69              \\
    \hline
    4              & 0.65              \\
    \hline
    5              & 0.67              \\
    \hline
    6              & 0.58              \\
    \hline
    7              & 0.61              \\
    \hline
    8              & 0.64              \\
    \hline
    9              & 0.63              \\
    \hline
    10             & 0.68              \\
    \hline
    11             & 0.61              \\
    \hline
    12             & 0.63              \\
    \hline
    13             & 0.67              \\
    \hline
    14             & 0.64              \\
    \hline
    15             & 0.65              \\
    \hline
    16             & 0.68              \\
    \hline
    17             & 0.66              \\
    \hline
    18             & 0.66              \\
    \hline
    19             & 0.56              \\
    \hline
    20             & 0.66              \\
    \hline
  \end{tabular}
\end{center}

\end{document}