\documentclass[11pt, letterpaper, includehead]{article}

%%%%%%%%%%%%%%%%%%%%% Pre-document %%%%%%%%%%%%%%%%%%%%%
\usepackage{fancyhdr}  % Allow for headers
\usepackage{graphicx}  % Allow for figures 
\usepackage{float}     % Allow for figure inserted in specified location
\usepackage{amsmath}   % Allow for aligned math
\usepackage{array}     % Allow for cell width manipulation
\usepackage{nicematrix}
\usepackage{amssymb} % Uhhhh what was this????
\usepackage{multicol}

\setlength{\parindent}{0pt} % Remove auto paragraph indents

% Get rid of those big ass margins
\usepackage[margin=1in]{geometry}

% Table cell formatting
\setlength{\arrayrulewidth}{0.25mm}
\setlength{\tabcolsep}{11pt}
\renewcommand{\arraystretch}{1.2}

\begin{document}

%%%%%%%%%%%%%%%%%%%%% Title Page %%%%%%%%%%%%%%%%%%%%%
\begin{titlepage}
  \begin{center}
    \Huge{\textbf{Lab 8}}\\
    \Huge{Conservation of Energy}
    \vfill
    \begin{figure}[H] % H makes the figure insert at the position in the document
      \centering 
      \includegraphics[width=6cm]{../logo.png}
    \end{figure}
    \large{\textbf{your name here}}\\
    \large{Julian Barossi, Liam Gilligan, Stephanie L'Heureux}\\
    \vspace{0.5cm}
    \normalsize
    \today
  \end{center}
\end{titlepage}

%%%%%%%%%%%%%%%%%%%%% TABLE OF CONTENTS %%%%%%%%%%%%%%%%%%%%%
\tableofcontents
\pagebreak % Move to next page

% Add a nice fancy header
\pagestyle{fancy}
\fancyhead{}
\fancyhead[C]{\textbf{Lab 8:} Conservation of Energy}

\section{Mass-spring system} % 1
\subsection{Unstretched position and spring constant k}
\subsubsection{Unstretched position}
The unstreched length of the spring is defined as the spring's natural length when no forces
are applied to stretch or compress it. We recorded the position using the digits display 
of the motion sensor and found the value of the initial position ($x_0$) to be $0.890m$

\subsubsection{Spring constant k}
The amount of force required to stretch or compress a spring a certain distance is defined by the spring constant $k$.
To find the value of $k$ we first measured the unstretched position of the spring
($x_0$) as discussed above. Then we added a hanger and mass totaling $100g$ to the spring and recorded the position
with the motion sensor. $x$ was found to be $0.915m$, and thus $\Delta x$ is $0.025 m$.\\

Because the hanger is at rest, the acceleration is zero, and thus the sum of the forces in the y
direction is zero. The weight of the hanging mass $mg$ acts downward and the force of the spring
($F_{sp}$) acts upward. In addition, note the force a spring exerts is defined by Hooke's law as $F_{sp} = -k \Delta x$. 
Using this information, $k$ is computed with the following calculations:
\begin{multicols}{2}
$$\Sigma F_y = ma_y$$
$$\Sigma F_y = 0$$
$$F_{sp} - mg = 0$$
$$F_{sp} = mg$$
$$F_{sp} = (0.100kg)(9.80m/s)$$
$$\boxed{F_{sp} = 0.980 N}$$

\columnbreak

$$F_{sp} = -k \Delta x$$
$$k = -\frac{F_{sp}}{\Delta x}$$
$$k = -\frac{0.980 N}{-0.025m}$$
$$k = \approx$$
$$\boxed{k = }$$
\end{multicols}

\subsection{Setup and data taking}
\begin{center} 
  \begin{tabular}{| m{3cm} |  m{3cm} |} 
    \hline
     \textbf{Quantity} & \textbf{Value}\\ 
      \hline
      $x_0$ & 8 \\ 
      \hline
      $k$ & $114.4 N/m$ \\ 
      \hline
      $m$ & $0.100kg$\\
      \hline
      $g$ & $9.8 m/s^2$\\
      \hline
  \end{tabular} 
\end{center}

\subsection{Data analysis}
\textbf{Focus on one complete cycle of the mass-spring system and indicate 
on your plot the two turning points (lower and higher) and the equilibrium point.}\\

\textbf{Is Ug is a minimum, a maximum, or neither? Explain why.}\\
\textbf{Is Usp is a minimum, a maximum, or neither? Explain why.}\\
\textbf{Is K is a minimum, a maximum, or neither? Explain why.}\\

\textbf{Why does the kinetic energy curve peak twice per cycle?}\\
\textbf{Is mechanical energy conserved? Justify your answer with 
data. If not, can you identify (and possibly eliminate) systematic 
errors in your data?}\\

\section{Conservation of energy}
\subsection{Introduction}
\subsection{Procedure}
\subsubsection{Equipment}

\begin{enumerate} 
  \item Smart car
  \item Motion sensor
  \item Force sensor
  \item Compressible spring
\end{enumerate}
\subsubsection{Steps}
\begin{enumerate} 
  \item Find the spring constant of the spring 
  \item Position the track at an incline with the spring launcher at the bottom 
        and the motion sensor at the top. Measure the angle of the incline.
  \item Predict the distance the car will travel up the ramp using conservation of energy to find $\Delta x_{thy}$
  \item Run the experiment and collect 10 data points of $\Delta x_{exp}$
\end{enumerate}
\subsection{Data}
\subsection{Data analysis}
\subsection{Conclusion}

\end{document}