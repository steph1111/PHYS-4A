\documentclass[11pt, letterpaper, includehead]{article}

%%%%%%%%%%%%%%%%%%%%% Pre-document %%%%%%%%%%%%%%%%%%%%%
\usepackage{fancyhdr}  % Allow for headers
\usepackage{graphicx}  % Allow for figures 
\usepackage{float}     % Allow for figure inserted in specified location
\usepackage{xcolor}

\setlength{\parindent}{0pt} % Remove auto paragraph indents

% Get rid of those big ass margins
\usepackage[margin=1in]{geometry}

\begin{document}
  %%%%%%%%%%%%%%%%%%%%% Title Page %%%%%%%%%%%%%%%%%%%%%
  \begin{titlepage} 
    \begin{center}
      \Huge{\textbf{Lab 1}}\\
      \Huge{Motion diagrams}
      \vfill
      \large{\textbf{Rectangle Repulsed Researchers}}\\
      \large{Julian Barossi, Liam Gilligan, Stephanie L'Heureux}\\
      \vspace{0.5cm}
      \normalsize
      \today
    \end{center}
  \end{titlepage}

  %%%%%%%%%%%%%%%%%%%%% TABLE OF CONTENTS %%%%%%%%%%%%%%%%%%%%%
  \tableofcontents
  \pagebreak % Move to next page


  % Add a nice fancy header
  \pagestyle{fancy}
  \fancyhead{}
  \fancyhead[C]{\textbf{Lab 1:} Motion diagrams}

  
  \setcounter{section}{2} % Make numbers start at 3
  \section{Matching Position vs. Time and Velocity vs. Time Graphs}

  %%%%%%%%%%%%%%%%%%%%% GRAPH 2 %%%%%%%%%%%%%%%%%%%%%
  \subsection{Graph 1}

  \begin{figure}[H] % H makes the figure insert at the position in the document
    \centering 
    \includegraphics[width=\linewidth]{graph_1.png}
  \end{figure}

  \subsubsection{Graph analysis}
  \emph{Graph 1} describes the position vs. time graph of a collision.
  The graph directly conveys that initially the object's position from 
  the sensor increases, then at $t\approx2.4s$ it's position begins decreasing. 
  The derivative of a position vs. time graph gives the velocity. From the slope 
  of the graph, one can conclude that the velocity over the interval $t\epsilon[0s, 2.4s)$
  is constant and positive. Then from $t \epsilon (2.4s, 6s]$ the velocity is constant 
  and negative. With this information, one can deduce the object of interest was given 
  an initial velocity and began moving in the positive direction, then it encountered 
  a barricade and reversed direction.
  $$\forall \, t \in [0s, 2.4s); \, v(t) = \frac{d}{dt}[s(t)] \approx \frac{0.95m - 0.6m}{2s - 1s} = 0.35 \, m/s$$
  $$\forall \, t \in (2.4s, 6s]; \, v(t) = \frac{d}{dt}[s(t)] \approx \frac{0.95m - 0.7m}{3s - 4s} = -0.25 \, m/s$$

  \subsubsection{Setup}
  To reproduce \emph{Graph 1}, we utilized a level track with a magnetic bumper mounted $60cm$ 
  from the \emph{PASCO Motion Sensor II}. With the cart positioned near the sensor, and we 
  pushed and released it towards the bumper. Upon impact with the bumper, it reversed 
  direction. The motion captured by the sensor was displayed using the \emph{PASCO Universal 
  850 Interface}. We consider friction and drag negligible in this application. \\

  %%%%%%%%%%%%%%%%%%%%% GRAPH 2 %%%%%%%%%%%%%%%%%%%%%
  \subsection{Graph 2}

  % Centered figure in text location
  \begin{figure}[H] % H makes the figure insert at the position in the document
    \centering 
    \includegraphics[width=\linewidth]{graph_2.png}
  \end{figure}

  \subsubsection{Graph analysis}
  \emph{Graph 2} depicts an object whose speed is increasing at a constant rate. This deduction is 
  justified by graphical analysis. The position increases over the interval 
  $t\epsilon[0s, 4.3s)$ at a constant rate, resembling a parabolic curve. The derivative of the 
  position vs. time graph yields velocity, which for this plot is increasing and positive. One 
  possible senerio that would recreate the graph is an object which accelerates down an inclined plane.\\
  
  \subsubsection{Setup}
  To replicate \emph{Graph 2}, we positioned a track such that it had a downward slope with the 
  \emph{PASCO Motion Sensor II} at the bottom. We released the cart from the top of the incline 
  and let it slide towards the sensor. The \emph{PASCO Universal 850 Interface} was used 
  to plot motion data collected with sonar. We consider friction and drag negligible in 
  this application.
  
  %%%%%%%%%%%%%%%%%%%%% GRAPH 3 %%%%%%%%%%%%%%%%%%%%%
  \subsection{Graph 3}

  \begin{figure}[H] % H makes the figure insert at the position in the document
    \centering 
    \includegraphics[width=\linewidth]{graph_3.png}
  \end{figure}

  \subsubsection{Graph analysis}
  Graph three depicts an object in freefall that hits the ground and stays there. We know this because the
  object's position starts from an increased height\\
  
  \subsubsection{Setup}
  Our setup consisted of a the sensor positioned updward. We dropped a large beach ball.\\

  %%%%%%%%%%%%%%%%%%%%% GRAPH 4 %%%%%%%%%%%%%%%%%%%%%
  \subsection{Graph 4}

  \begin{figure}[H] % H makes the figure insert at the position in the document
    \centering 
    \includegraphics[width=\linewidth]{graph_4.png}
  \end{figure}

  \subsubsection{Graph analysis}
  \emph{Graph 4} depicts an object moving down an incline. It comes in contact with somethinhg at the
  bottom and changes direction. As it changes direction it slows down. Then at $t=2.6s$ its velocity 
  becomes negaiive but the acceleration staus positive where it is slowing down\\
  $$\forall \, t \in [0s, 2.6s); \, a(t) = \frac{d}{dt}[v(t)] \approx \frac{0.6 \, m/s - 0.2 \, m/s}{2s - 1s} = 0.4 \, m/s^2$$
  $$\forall \, t \in (2.6s, 4.1s]; \, a(t) = \frac{d}{dt}[v(t)] \approx \frac{-0.5 \, m/s -(-0.4 \, m/s)}{2.6s - 2.8s} = 0.3 \, m/s^2$$ % Should be 0.4
  
  \subsubsection{Setup}
  Our setup consisted of a the sensor positioned at the biotton of an incline plane. We released the cart from the top 

  %%%%%%%%%%%%%%%%%%%%% GRAPH 5 %%%%%%%%%%%%%%%%%%%%%
  \subsection{Graph 5}

  \begin{figure}[H] % H makes the figure insert at the position in the document
    \centering 
    \includegraphics[width=\linewidth]{graph_5.png}
  \end{figure}

  \subsubsection{Graph Analysis}
  Graph 5 depicts an object that starts from rest then is that ts pushed then slows to a stop. 
  We know this because the graph depicts the initial velocity as $0m/s$, then the velocity rapidly 
  increases and the acceleration is posiyive. It the acceleratyion goes to zero and he velocity is constant
  $$\forall \, t \in (2.4s, 2.8s); \, a(t) = \frac{d}{dt}[v(t)] \approx \frac{0.1 \, m/s - 0.24 \, m/s}{2.6s - 2.7s} = 1.4 \, m/s^2$$


  \subsubsection{Setup}
  Our setup consisted a flat plane and pushed the cart away from the sensor


  %%%%%%%%%%%%%%%%%%%%% GRAPH 6 %%%%%%%%%%%%%%%%%%%%%
  \subsection{Graph 6}

  \begin{figure}[H] % H makes the figure insert at the position in the document
    \centering 
    \includegraphics[width=\linewidth]{graph_6.png}
  \end{figure}

  \subsubsection{Graph Analysis}
  The graph above depicts an object being sped up rapidly before decelerating
  at a constant rate.
  $$\forall \, t \in (2, 2.5s); \, a(t) = \frac{d}{dt}[v(t)] \approx \frac{0.08 \, m/s - 0.38 \, m/s}{2.1s - 2.3s} = 1.5 \, m/s^2$$
  $$\forall \, t \in (2.6s, 4.1s]; \, a(t) = \frac{d}{dt}[v(t)] \approx \frac{0.6 \, m/s -(-0.4 \, m/s)}{2.8s - 4.8s} = -0.5 \, m/s^2$$ 
  % Graph 6: ~ −0.53m/s/s
  \subsubsection{Setup}
  To recreate the motion described by the graph, we decided to use an inclined plane.
  The graph initially shows a positive acceleration, but later a constant negative acceleration.
  To recreate this on the inclined plane, we pushed the cart up, then let
  gravity slow it down and accelerate it back down the incline.

  %%%%%%%%%%%%%%%%%%%%% GRAPH 7 %%%%%%%%%%%%%%%%%%%%%

  \subsection{Graph 7}

  \begin{figure}[H] % H makes the figure insert at the position in the document
    \centering 
    \includegraphics[width=\linewidth]{graph_7.png}
  \end{figure}

  \subsubsection{Graph Analysis}
  Graph 7 depicts some sort of oscillating motion. Because the graph is 
  sinusoidal, we know that the graph's position and acceleration will look
  just like the velocity, just shifted by some amount. This means that the 
  position is osciallting, as is the velocity and the acceleration.
  \subsubsection{Setup}
  In order to achieve the oscillating motion displayed, we decided to attach
  a weight to a spring. If we stretched the spring by some amount, then
  let go as we started recording the velocity, we knew that we could capture
  the same motion described by the graph.




  % \centerline{\textbf{Graph 1}}\\
  % % Any kind of straight lines, calculate a slope. D
  % \textbf{Graph 2}\\
  % \textbf{Graph 3}\\
  % \textbf{Graph 4}\\
  % \textbf{Graph 5}\\
  % \textbf{Graph 6}\\
  % \textbf{Graph 7}\\

  % \centerline{}\includegraphics{graph_1.png}}

\end{document}

