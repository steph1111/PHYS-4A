\documentclass[11pt, letterpaper, includehead]{article}

%%%%%%%%%%%%%%%%%%%%% Pre-document %%%%%%%%%%%%%%%%%%%%%
\usepackage{fancyhdr}  % Allow for headers
\usepackage{graphicx}  % Allow for figures 
\usepackage{float}     % Allow for figure inserted in specified location
\usepackage{xcolor}
\usepackage{amsmath}

\setlength{\parindent}{0pt} % Remove auto paragraph indents

% Get rid of those big ass margins
\usepackage[margin=1in]{geometry}

\begin{document}
  %%%%%%%%%%%%%%%%%%%%% Title Page %%%%%%%%%%%%%%%%%%%%%
  \begin{titlepage} 
    \begin{center}
      \Huge{\textbf{Lab 2}}\\
      \Huge{Experimental Uncertainty}
      \vfill
      \large{\textbf{Rectangle Repulsed Researchers}}\\
      \large{Julian Barossi, Liam Gilligan, Stephanie L'Heureux}\\
      \vspace{0.5cm}
      \normalsize
      \today
    \end{center}
  \end{titlepage}

  %%%%%%%%%%%%%%%%%%%%% TABLE OF CONTENTS %%%%%%%%%%%%%%%%%%%%%
  \tableofcontents
  \pagebreak % Move to next page


  % Add a nice fancy header
  \pagestyle{fancy}
  \fancyhead{}
  \fancyhead[C]{\textbf{Lab 2:} Experimental Uncertainty}

  \section{Measuring the Time of a Dropped Pencil} % 1
  
  \subsection{Predicted time $t_{thy}$ for the pencil to fall from rest $2.0m$} % 1.1

  \begin{align*}
    v_0t + \frac{1}{2}at^2 &= x_f - x_0\\
    \frac{1}{2}at^2 &= -x_0\\
    t^2 &= \frac{-2x_0}{a}\\
    t &= \sqrt{\frac{-2x_0}{a}}\\
    t &= \sqrt{\frac{-2(2.0m)}{-9.8m/s^2}}\\
    t &= \sqrt{\frac{20}{49}}s\\
    t &\approx 0.64s
  \end{align*}

  \subsection{Recorded time for each trial} % 1.2
  text
  \begin{center} 
    \begin{tabular}{|| c | c ||} 
      \hline
      Trial & Value \\
      \hline
        1 & 0.55 \\ 
         \hline
        2 & 0.69 \\ 
         \hline
        3 & 0.69 \\ 
         \hline
        4 & 0.65 \\ 
         \hline
        5 & 0.67 \\ 
         \hline
        6 & 0.58 \\ 
         \hline
        7 & 0.61 \\ 
         \hline
        8 & 0.64 \\ 
         \hline
        9 & 0.63 \\ 
         \hline
        10 & 0.68 \\ 
         \hline
        11 & 0.61 \\ 
         \hline
        12 & 0.63 \\ 
         \hline
        13 & 0.67 \\ 
         \hline
        14 & 0.64 \\ 
         \hline
        15 & 0.65 \\ 
         \hline
        16 & 0.68 \\ 
         \hline
        17 & 0.66 \\ 
         \hline
        18 & 0.66 \\ 
         \hline
        19 & 0.56 \\ 
         \hline
        20 & 0.66 \\ 
         \hline
      \hline 
    \end{tabular} 
  \end{center}
 
 
   

  \setcounter{subsection}{3} % Make next section start at 1.4
  \subsection{Mean average time ($\bar{t}$) for 100 measurements} % 1.4

  \subsection{Standard deviation ($\sigma$) for 100 measurements} % 1.5

  \subsection{Approximately $68\%$ of the values of any measurement should fall within one 
  standard deviation ($1 \sigma$) of the mean value ($\bar{t}$). Therefore $68\%$ of measured 
  values should be $\geq (t - \sigma_t)$ and $\leq (t + \sigma_t)$} % 1.6 

  \subsubsection{Percentage of values which fall in one standard deviation 
  ($1\sigma$) of the mean ($\bar{t}$)}

  \subsubsection{Does the randomly distributed data match up with the statsitical approximation
  that $68\%$ should fall within one standard deviation ($1\sigma$) of the mean ($\bar{t}$)}

  \subsection{Approximately $95\%$ of the values of any measurement should fall within two 
  standard deviations ($2 \sigma$) of the mean value ($\bar{t}$). Therefore $95\%$ of measured 
  values should be $\geq (t - 2 \sigma_t)$ and $\leq (t + 2 \sigma_t)$} % 1.7 

  \subsubsection{Percentage of values which fall in one standard deviation 
  ($1\sigma$) of the mean ($\bar{t}$)}

  \subsubsection{Does the randomly distributed data match up with the statsitical approximation
  that $68\%$ should fall within one standard deviation ($1\sigma$) of the mean ($\bar{t}$)}



\end{document}

